%% LaTeX2e file `Thesis-businessrulesbewertung.tex'
%% generated by the `filecontents' environment
%% from source `Thesis' on 2018/02/21.
%%
 \begin{longtable}{P{4cm}|X}
  \caption{Umsetzung der IoT spezifischen Anforderungen durch eEPK}\\
  \label{table:evaluierungskriterien}
  % Definition des ersten Tabellenkopfes auf der ersten Seite
  \textbf{Anforderung} & \textbf{Erklärung}   \\ \hline
  \endfirsthead % Erster Kopf zu Ende
  %  Definition des Tabellenkopfes auf den folgenden Seiten
  \textbf{Anforderung} & \textbf{Erklärung}  \\ \hline
  \endhead % Zweiter Kopf ist zu Ende

  % Ab hier kommt der Inhalt der Tabelle
  Device als Akteur & Device kann als Stelle der Zuständigkeit darstellen lassen. \\ \hline
  Physische Dinge  & Physisches Ding kann sich als Informationsquelle in \ac{eepk} darstellen lassen allerings kann der Zustand von Informationsquellen nicht verändert werden.\\ \hline
  Spezielle Aufgabentypen & \ac{eepk} besitzt für die Darstellung von Aufgaben nur das Modell der Funktion.\\ \hline
  Informationen  & Informationen besitzen ihr eigenes Symbol in \ac{eepk} allerdings wird dieses bereits für die Modellierung des physischen Dings verwendet was zu Komplikationen bei der Interpretation des Modells führen kann. Des Weiteren lassen sich die \ac{iot} spezifischen Anforderungen an die Informationen nicht darstellen.\\ \hline
  Mobiltät & In \ac{eepk} Durch das Eintreten bestimmter Ereignisse lässt sich die Ortsabhängigkeit von Aufgaben abbilden. Die Mobilität von den \ac{eoi} selbst lässt sich allerdings nicht darstellen.\\ \hline
  Granularität & Eine Aufteilung eines Prozesses in mehrere Unterprozesse ist nicht möglich.\\
 \end{longtable}
