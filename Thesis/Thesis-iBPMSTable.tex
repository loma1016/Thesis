%% LaTeX2e file `Thesis-iBPMSTable.tex'
%% generated by the `filecontents' environment
%% from source `Thesis' on 2018/02/07.
%%
 \begin{longtable}{p{4.5cm}X}
  \caption{Kernfunktionalitäten eines iBPMS \cite{ibpms}}\\
  \label{table:GartnerIBPMS}
  % Definition des ersten Tabellenkopfes auf der ersten Seite
  \textbf{Funktionalität} & \textbf{Erklärung}  \\
  \hline
  \endfirsthead % Erster Kopf zu Ende
  %  Definition des Tabellenkopfes auf den folgenden Seiten
  \textbf{Funktionalität} & \textbf{Erklärung}  \\
  \hline
  \endhead % Zweiter Kopf ist zu Ende
  Weiter auf der n{\"a}chste Seite\\
  \endfoot
  \hline
  Tabelle zu Ende \\
  \endlastfoot
  % Ab hier kommt der Inhalt der Tabelle
  Interaktionsmanagement & Die Fähigkeit, verschiedene Arten von Aktivitäten und Interaktionen zur Laufzeit zu orchestrieren, um die Arbeit zu unterstützen, die Menschen, Systeme und "Dinge" (wie im IoT) leisten, um spezifische Geschäftsergebnisse zu erzielen
  \\ \hline
  Hochproduktive App Generierung & Ermöglicht  IT-Entwicklern, schnell und einfach eine prozessorientierte Anwendung zu erstellen.
  Anwendungen, die auf der Plattform basieren, verwenden ein Metadatenmodell, um den gesamten Lebenszyklus von Geschäftsprozessen zu verwalten und prozessbezogene Daten zu manipulieren
  \\ \hline
  Überwachung und Geschäftsanpassung & iBPMS Plattformen unterstützen Business Activity Monitoring um den Status von Prozessinstanzen, Fällen und anderen Verhaltensweisen in nahezu Echtzeit kontinuierlich zu verfolgen.
  \\ \hline
  Regeln und Entscheidungs Management &  Softwaresysteme - wie inference engines, recommendation engines und decision management capabilities -, die als Orientierungshilfe dienen, um menschliche oder automatisierte betriebliche Entscheidungen nach Geschäftsrichtlinien zu treffen
  \\ \hline
  Analysen & Wendet Logik und Statistiken auf Daten an, um Erkenntnisse für bessere Entscheidungen zu gewinnen. Ein \ac{ibpms} kann prädiktive Analysen wie z.B. Scoring Services oder präskriptive Analysen wie z.B. optimization engines enthalten oder mit diesen in Verbindung stehen
  \\ \hline
  Kompatibilität &  Kompatibilität mit externen Anwendungsdiensten und Systemen, die ein \ac{ibpms}-Adapter und Adapter-Entwicklungswerkzeuge ermöglichen. Zu diesen Diensten und Systemen gehören benutzerdefinierte und kommerzielle Standardanwendungen sowie Cloud-basierte SaaS-Anwendungen und deren Datenbanken.
  \\ \hline
  Mobile Verwendbarkeit & Die Möglichkeit, von einer Vielzahl mobiler Geräte, einschließlich Smartphones und Tablets, auf Anwendungen zuzugreifen. Die Plattform bietet nicht nur Zugriff von jedem Ort aus, sondern optimiert auch die nativen Fähigkeiten des Mobilgeräts, einschließlich der Kamera und anderer Sensoren
  \\ \hline
  Kontext- und Verhaltensstatistik & Die Fähigkeit der Plattform verkürzt die Zeit, welche benötigt wird, um Verhaltensweisen die zur Verbesserung der Geschäftsergebnisse erforderlich sind zu erkennen und zu optimieren. Dies kann die Analyse der vergangenen Ausführungshistorie oder die Simulation von Verhaltensvorschlägen beinhalten.
  \\ \hline
 \end{longtable}
