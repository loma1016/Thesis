
\documentclass[a4paper, 12pt, twoside, headsepline=true]{scrartcl} % headsepline ist für die Linie unter der Kopfzeile verantwortlich

% Package für Sprachformatierung und Spellchecks
% http://ctan.space-pro.be/tex-archive/macros/latex/contrib/polyglossia/polyglossia.pdf
\usepackage{polyglossia}
\setdefaultlanguage[spelling=new]{german}

% Package zum Ändern von Headern & Footers
% mirrors.ctan.org/macros/latex/contrib/koma-script/doc/scrguien.pdf
\usepackage[automark]{scrlayer-scrpage}
\clearpairofpagestyles
\ihead{\headmark}
\ohead{\pagemark}

\pagestyle{scrheadings}
\setkomafont{pageheadfoot}{\small}

% Package zum Festlegen von Zeilenabstand
% https://www.namsu.de/Extra/pakete/Setspace.html
\usepackage[onehalfspacing]{setspace}
\setmainfont{Charis SIL} % set the main body font (\textrm), assumes Charis SIL is installed
\setlength{\parindent}{0pt}


\usepackage{csquotes}

% Package für das Referenzieren von Quellen aus .bib-Datei
\usepackage[style=ieee]{biblatex}
\addbibresource{mybib.bib}

% Package zum "Patchen" von Befehlen
% http://vesta.informatik.rwth-aachen.de/ftp/pub/mirror/ctan/macros/latex/contrib/xpatch/xpatch.pdf
% Überschreibt biblatex-Funktionalität, damit keine leeren Jahresklammern gedruckt werden
\usepackage{xpatch}
\xpatchbibdriver{online}
{\printtext[parens]{\usebibmacro{date}}}
{\iffieldundef{year}
	{}
	{\printtext[parens]{\usebibmacro{date}}}}
{}
{\typeout{There was an error patching biblatex-ieee (specifically, ieee.bbx's @online driver)}}

% Package zum Anpassen von enumerations-items 
% https://de.wikibooks.org/wiki/LaTeX-W%C3%B6rterbuch:_enumitem
\usepackage{enumitem}
\renewcommand{\labelitemiii}{$\star$}

% Package für die Anpassung der Seitengestaltung, z.B. Seitenränder
% https://www.namsu.de/Extra/pakete/Geometry.html
\usepackage{geometry}

% Package für Tabellengestaltung mit horizontalen Trennlinien
% https://www.namsu.de/Extra/pakete/Booktabs.html
\usepackage{booktabs}

% Package für Gestaltung und Anpassung einzelner Tabellenzellen 
% http://babyname.tips/mirrors/ctan/macros/latex/contrib/makecell/makecell.pdf
\usepackage{makecell}

% Package für Positionierung von "floats", also nichtfesten Elementen wie Tabellen oder Bildern, die sich je nach Textgestaltung verschieden auf der Seite befinden müssen
% http://vesta.informatik.rwth-aachen.de/ftp/pub/mirror/ctan/macros/latex/contrib/float/float.pdf
\usepackage{float}

% Package für Tabellengestaltung 
% https://www.namsu.de/Extra/pakete/Tabularx.html
\usepackage{tabularx,ragged2e}
\newcolumntype{L}{>{\RaggedRight\arraybackslash}X}

% Package zum Anpassen von Macros/KeyValue-Daten. Hab ich garnicht verwendet, glaube ich
% http://mirrors.ibiblio.org/CTAN/macros/latex/contrib/adjustbox/adjustbox.pdf
\usepackage[export]{adjustbox}

% Package für Verwaltung von Acronymen 
% https://www.namsu.de/Extra/pakete/Acronym.html
\usepackage[printonlyused]{acronym}

% Package zum Implementieren von Textlinks
% https://www.namsu.de/Extra/pakete/Hyperref.html
\usepackage[
german,
colorlinks=true,
linkcolor=blue, % einfache interne Verknüpfungen
%TODO: Folgende Zeile für Druckdokument verwenden!
%hidelinks=true
anchorcolor=black,% Ankertext
citecolor=green, % Verweise auf Literaturverzeichniseinträge im Text
urlcolor=cyan % Farbe der URLs
 %Back-Links zu den Kapiteln
]{hyperref}
\apptocmd{\UrlBreaks}{\do\f\do\m}{}{}
\setcounter{biburllcpenalty}{9000}% Kleinbuchstaben
\setcounter{biburlucpenalty}{9000}% Großbuchstaben
\setcounter{biburlnumpenalty}{9000}% Zahlen
%\usepackage[german,draft]{hyperref}

%Neudefinition für den Name von Subsubsection, damit "Unterunterabschnitt" nicht bei Referenzen auftaucht
\def\subsubsectionname{Unterabschnitt}%
\def\subsubsectionautorefname{Unterabschnitt}%

% Package für besseren Umgang mit Grafiken
% http://ftp.gwdg.de/pub/ctan/macros/latex/required/graphics/grfguide.pdf
\usepackage{graphicx}

% Package für Platzieren von Bildern neben/im Text
% https://www.namsu.de/Extra/pakete/Wrapfig.html
\usepackage{wrapfig}
\newcommand{\myhline}{\noalign{\global\arrayrulewidth0,1cm}\hline
	\noalign{\global\arrayrulewidth1pt}}
% Umbenennung der Caption Beschreibung unter Bildern von "Abbildung" in "Abb." um Platz zu sparen.
\addto\captionsgerman{%
	\renewcommand{\figurename}{Abb.}%
}

% Definiere eine neue Liste bei der die Unterpunkte auch nummeriert sind, wie bei einem Inhaltsverzeichnis.
\newlist{legal}{enumerate}{10}
\setlist[legal]{label*=\arabic*.}
%\setcounter{tocdepth}{4}
%\setcounter{secnumdepth}{4}

\makeatletter 
\@addtoreset{figure}{section} 
\@addtoreset{table}{section} 
\makeatother 

%Commands zum Neustarten der Seitennummerierung ab Inahltsverzeichnis, falls vorher noch Titelseiten o.Ä. folgen
\renewcommand{\thefigure}{\thesection.\arabic{figure}} 
\renewcommand{\thetable}{\thesection.\arabic{table}} 

\begin{document}
% Nötig um in der PDF Datei einen Lesezeicheneintrag für das Inhaltsverzeichnis zu bekommen.
\pdfbookmark[1]{Inhaltsverzeichnis}{toc}
\tableofcontents
\clearpage
\newpage

\addsec{Abkürzungsverzeichnis}

\pdfbookmark[2]{Abkürzungsverzeichnis}{toc}
% Angabe in eckigen Klammern sollte das längste Acronym enthalten.
% Das ist notwendig damit sich der Einschub am längsten Acronym orientiert.
\begin{acronym}[header=Abkürzungsverzeichnis]
	\acro{aclabel}[BSP]{Beispiel}
\end{acronym}

\clearpage

\addcontentsline{toc}{section}{Tabellenverzeichnis}
\listoftables
\clearpage

\addcontentsline{toc}{section}{Abbildungsverzeichnis}
\listoffigures
\clearpage


\section{Einleitung} \label{sec:section}
Verwendeter Editor LaTeX: TeXstudio, mit Compiler XeLaTeX

\subsection{Motivation} \label{sec:subsection}
Das Internet of Things, kurz IoT ist eines der größten IT-Buzzwords
der letzten Jahre. Die durch die Vernetzung von Gegenständen
gewonnen Ereignisse bzw. Daten bieten neben dem Potential der
Prozessoptimierung und Erweiterung noch die Möglichkeit zur
Generierung völlig neuer Geschäftsprozesse und Modelle. Des
Weiteren sinken die Kosten dafür physische Dinge mit Sensoren
auszustatten und untereinander zu vernetzen, was zu einem hohen
Andrang an IoT Projekten führt[1]. Laut Gartner sollen im Jahr 2020
mehr als die Hälfte der wichtigsten Geschäftsprozess Elemente des IoT
beinhalten[1]. 

\subsection{Problemstellung} 
Häufig gestaltet sich die Darstellung und Modellierung
der neuen Geschäftsprozesse jedoch schwierig, da Standards wie
BPMN nur bedingt hierfür geeignete Elemente vorsehen. Erschwert
wird dies dadurch, dass keine klare Abgrenzung zwischen dem
eigentlichen Geschäftsprozess und dem Sammeln, Aggregieren und
Auswerten der Daten besteht. Diese Fragestellungen bilden die
Grundlage für diese Thesis

\subsection{Zielsetzung}
Ziel der Thesis ist die Konzeption eines Modellierungsansatzes für IoT
Workflows.
Zunächst werden Unterscheidungsmerkmale zwischen IoT und
normalen Workflows herausgearbeitet. Daraus werden
generalisierte IoT Workflows festgelegt, welche typische Muster und
Best Practices beinhalten. Anhand der generalisierten IoT Workflows
werden Unterschiede und Besonderheiten gegenüber Workflows
herausgearbeitet, welche bei der Modellierung berücksichtigt werden
müssen. Aus diesen werden Kriterien abgeleitet welche zur
Evaluierung der Modellierungsmöglichkeiten verwendet werden.
Anschließend werden gängige Modellierungsmöglichkeiten vorgestellt
und anhand der vorher festgelegten Kriterien evaluiert.
Der so konzipierte Ansatz wird auf ein oder mehrere Use-Cases
angewandt, abschließend erfolgt eine Bewertung.

\subsection{Aufbau der Thesis}

\newpage

\section{Grundlagen} \label{sec:section2}

\subsection{Prozess Modellierung}

\subsubsection{BPMN}

\subsubsection{UML}

\subsubsection{Geschäftsregeln}

\subsubsection{IoT - A}

\subsection{IoT}

\subsection{BPM}

\newpage


\section{IoT Worklfows}

\subsection{Typische Muster und Best Practices von IoT Workflows}

\subsection{Unterschiede IoT Workflows zu Workflows}

\subsection{Evaluierungskriterien}

\subsection{Bewertung der Modellierungsmethoden}

\subsection{ Modellierungs -Konzept}

\newpage

\section{Evaluierung}

\newpage

\section{Schlussteil}

\subsection{Ergebnis}

\subsection{Fazit}

\subsection{Weiterführende Arbeit/ Ausblick}

\newpage

\addsec{Literaturverzeichnis}
\printbibliography[heading=none]
\newpage
\addsec{Anhang}
\subsection*{Unterbereich Anhang}
\end{document}
