
\documentclass[a4paper, 12pt, twoside, headsepline=true]{scrartcl} % headsepline ist für die Linie unter der Kopfzeile verantwortlich

% Package für Sprachformatierung und Spellchecks
% http://ctan.space-pro.be/tex-archive/macros/latex/contrib/polyglossia/polyglossia.pdf
\usepackage{polyglossia}
\setdefaultlanguage[spelling=new]{german}

% Package zum Ändern von Headern & Footers
% mirrors.ctan.org/macros/latex/contrib/koma-script/doc/scrguien.pdf
\usepackage[automark]{scrlayer-scrpage}
\clearpairofpagestyles
\ihead{\headmark}
\ohead{\pagemark}

\pagestyle{scrheadings}
\setkomafont{pageheadfoot}{\small}

% Package zum Festlegen von Zeilenabstand
% https://www.namsu.de/Extra/pakete/Setspace.html
\usepackage[onehalfspacing]{setspace}
\setmainfont{Charis SIL} % set the main body font (\textrm), assumes Charis SIL is installed
\setlength{\parindent}{0pt}


\usepackage{csquotes}

% Package für das Referenzieren von Quellen aus .bib-Datei
\usepackage[style=ieee]{biblatex}
\addbibresource{mybib.bib}

% Package zum "Patchen" von Befehlen
% http://vesta.informatik.rwth-aachen.de/ftp/pub/mirror/ctan/macros/latex/contrib/xpatch/xpatch.pdf
% Überschreibt biblatex-Funktionalität, damit keine leeren Jahresklammern gedruckt werden
\usepackage{xpatch}
\xpatchbibdriver{online}
{\printtext[parens]{\usebibmacro{date}}}
{\iffieldundef{year}
	{}
	{\printtext[parens]{\usebibmacro{date}}}}
{}
{\typeout{There was an error patching biblatex-ieee (specifically, ieee.bbx's @online driver)}}

% Package zum Anpassen von enumerations-items 
% https://de.wikibooks.org/wiki/LaTeX-W%C3%B6rterbuch:_enumitem
\usepackage{enumitem}
\renewcommand{\labelitemiii}{$\star$}

% Package für die Anpassung der Seitengestaltung, z.B. Seitenränder
% https://www.namsu.de/Extra/pakete/Geometry.html
\usepackage{geometry}

% Package für Tabellengestaltung mit horizontalen Trennlinien
% https://www.namsu.de/Extra/pakete/Booktabs.html
\usepackage{booktabs}

% Package für Gestaltung und Anpassung einzelner Tabellenzellen 
% http://babyname.tips/mirrors/ctan/macros/latex/contrib/makecell/makecell.pdf
\usepackage{makecell}

% Package für Positionierung von "floats", also nichtfesten Elementen wie Tabellen oder Bildern, die sich je nach Textgestaltung verschieden auf der Seite befinden müssen
% http://vesta.informatik.rwth-aachen.de/ftp/pub/mirror/ctan/macros/latex/contrib/float/float.pdf
\usepackage{float}

% Package für Tabellengestaltung 
% https://www.namsu.de/Extra/pakete/Tabularx.html
\usepackage{tabularx,ragged2e}
\newcolumntype{L}{>{\RaggedRight\arraybackslash}X}

% Package zum Anpassen von Macros/KeyValue-Daten. Hab ich garnicht verwendet, glaube ich
% http://mirrors.ibiblio.org/CTAN/macros/latex/contrib/adjustbox/adjustbox.pdf
\usepackage[export]{adjustbox}

% Package für Verwaltung von Acronymen 
% https://www.namsu.de/Extra/pakete/Acronym.html
\usepackage[printonlyused]{acronym}

% Package zum Implementieren von Textlinks
% https://www.namsu.de/Extra/pakete/Hyperref.html
\usepackage[
german,
colorlinks=true,
linkcolor=blue, % einfache interne Verknüpfungen
%TODO: Folgende Zeile für Druckdokument verwenden!
%hidelinks=true
anchorcolor=black,% Ankertext
citecolor=green, % Verweise auf Literaturverzeichniseinträge im Text
urlcolor=cyan % Farbe der URLs
 %Back-Links zu den Kapiteln
]{hyperref}
\apptocmd{\UrlBreaks}{\do\f\do\m}{}{}
\setcounter{biburllcpenalty}{9000}% Kleinbuchstaben
\setcounter{biburlucpenalty}{9000}% Großbuchstaben
\setcounter{biburlnumpenalty}{9000}% Zahlen
%\usepackage[german,draft]{hyperref}

%Neudefinition für den Name von Subsubsection, damit "Unterunterabschnitt" nicht bei Referenzen auftaucht
\def\subsubsectionname{Unterabschnitt}%
\def\subsubsectionautorefname{Unterabschnitt}%

% Package für besseren Umgang mit Grafiken
% http://ftp.gwdg.de/pub/ctan/macros/latex/required/graphics/grfguide.pdf
\usepackage{graphicx}

% Package für Platzieren von Bildern neben/im Text
% https://www.namsu.de/Extra/pakete/Wrapfig.html
\usepackage{wrapfig}
\newcommand{\myhline}{\noalign{\global\arrayrulewidth0,1cm}\hline
	\noalign{\global\arrayrulewidth1pt}}
% Umbenennung der Caption Beschreibung unter Bildern von "Abbildung" in "Abb." um Platz zu sparen.
\addto\captionsgerman{%
	\renewcommand{\figurename}{Abb.}%
}

% Definiere eine neue Liste bei der die Unterpunkte auch nummeriert sind, wie bei einem Inhaltsverzeichnis.
\newlist{legal}{enumerate}{10}
\setlist[legal]{label*=\arabic*.}
%\setcounter{tocdepth}{4}
%\setcounter{secnumdepth}{4}

\makeatletter 
\@addtoreset{figure}{section} 
\@addtoreset{table}{section}
\makeatother 

%Commands zum Neustarten der Seitennummerierung ab Inahltsverzeichnis, falls vorher noch Titelseiten o.Ä. folgen
\renewcommand{\thefigure}{\thesection.\arabic{figure}} 
\renewcommand{\thetable}{\thesection.\arabic{table}} 

\begin{document}
% Nötig um in der PDF Datei einen Lesezeicheneintrag für das Inhaltsverzeichnis zu bekommen.
\pdfbookmark[1]{Inhaltsverzeichnis}{toc}
\tableofcontents
\clearpage
\newpage

\addsec{Abkürzungsverzeichnis}

\pdfbookmark[2]{Abkürzungsverzeichnis}{toc}
% Angabe in eckigen Klammern sollte das längste Acronym enthalten.
% Das ist notwendig damit sich der Einschub am längsten Acronym orientiert.
\begin{acronym}[header=Abkürzungsverzeichnis]
	\acro{bpd}[BPD]{Business Process Diagram}
	\acro{bpel}[BPEL]{Business Process Execution Language}
	\acro{bpm}[BPM]{Business Process Management}
	\acro{bpmn}[BPMN]{Business Process Model and Notation}
	\acro{bpmn4cps}[BPMN4CPS]{Business Process Model and Notation for Cyber-Physical Systems}
	\acro{cmmn}[CMMN]{Case Management Model and Notation}
	\acro{eoi}[EoI]{Entity of Interest}
	\acro{iot}[IoT]{Internet of Things}
	\acro{iota}[IoT-A]{Internet of Things - Architecture}
	\acro{omg}[OMG]{Object Management Group}
	\acro{uml}[UML]{Unified Modelling Language}
\end{acronym}

\clearpage

\addcontentsline{toc}{section}{Tabellenverzeichnis}

\listoftables

\clearpage

\addcontentsline{toc}{section}{Abbildungsverzeichnis}
\listoffigures
\clearpage


\section{Einleitung} \label{sec:section}

\subsection{Motivation} \label{sec:subsection}
Das \ac{iot} ist eines der größten IT-Buzzwords der letzten Jahre und und beschreibt die durch eingebettete Elektronik ermöglichte Vernetzung von physischen Dingen. Die dadurch gewonnen Ereignisse bzw. Daten bieten neben dem Potential der
Prozessoptimierung und Erweiterung noch die Möglichkeit zur Generierung völlig neuer Geschäftsprozesse und Modelle. 
Das Weiteren sinken die Kosten dafür physische Dinge mit Sensoren auszustatten und untereinander zu vernetzen, was zu einem hohen Andrang an \ac{iot} Projekten führt. Laut Gartner sollen im Jahr 2020 mehr als die Hälfte der wichtigsten Geschäftsprozess Elemente des \ac{iot} beinhalten.
%Quelle Gartner
 Da der Wettbewerb auf dem Technologiemarkt rasant zu nimmt, ist es unerlässlich  sich von der Konkurrenz abzuheben. Die Verwaltung von Smart Devices mit \ac{bpm} ermöglicht eine einfache Wartung ihrer Orchestrierung. Des weiteren bietet es auch den Aspekt der Nachverfolgung, die es ermöglicht KPIs über die Prozesse und Devices einfach zu ermitteln. Diese KPIs sind maßgebend für die Entscheidung, wie man Schwarm von Objekten am besten bedienen können sind.
%Quelle http://data-informed.com/bpm-of-things-the-next-generation-of-the-internet-of-things/

\subsection{Problemstellung} 
Häufig gestaltet sich die Darstellung und Modellierung der neuen Geschäftsprozesse jedoch schwierig, da Standards wie \ac{bpmn} nur bedingt hierfür geeignete Elemente vorsehen. Außerdem nimmt die Anzahl der vernetzten Geräte von Tag zu Tag zu und bietet somit mehr Raum für mangelhafte oder fehlende Kommunikation zwischen den Geräten. Dieser Umstand erschwert eine Verwaltung von IoT Workflows ohne \ac{bpm} wesentlich. Diese Problemstellungen bilden die Grundlage für diese Thesis.

\subsection{Zielsetzung}
Ziel der Thesis ist die Konzeption eines Modellierungsansatzes für \ac{iot}
Workflows. Hierfür werden Grundlegende Besonderheiten von \ac{iot} Workflows festgehalten und davon ausgehend Evaluierungskriterien für die Bewertung gängiger abgeleitet. Anhand der Kriterien werden Modellierungsmethoden bewertet und gegebenenfalls mögliche Erweiterungsmöglichkeiten vorgestellt. Der daraus resultierende Ansatz wird auf vorhandene Use-Cases angewandt und bewertet.

\subsection{Aufbau der Thesis}
Nach der Einleitung mit Motivation, Problemstellung, Zielsetzung
 sowie dem Aufbau der Thesis folgen Grundlagen im Bereich des \ac{iot}, der Prozess Modellierung, 
  des \ac{bpm}, der \ac{iota} sowie der \ac{bpmn4cps}, welche zum Verständnis der weiteren Arbeit dienen.
 
 Im Hauptteil werden typische Muster von \ac{iot} Workflows festgelegt. Aus den festgelegten Workflows werden Unterschiede und Besonderheiten zwischen \ac{iot} Workflows und Workflows ohne \ac{iot} Integration herausgearbeitet, welche bei der Modellierung zu berücksichtigen sind.
 Anhand der Unterschiede werden Evaluierungskriterien für die Geschäftsprozess Modellierung abgeleitet. Diese Evaluierungskriterien werden im Anschluss dazu verwendet um bestehende Modellierungsmethoden auf ihre Eignung zur Modellierung von \ac{iot} Workflows zu bewerten. Basierend auf der Bewertung wird ein Modellierungskonzept für \ac{iot} Workflows festgelegt. Infolge dessen werden ein oder mehrere Use-Cases analysiert und das Modellierungskonzept darauf angewandt. Anhand der Ergebnisse wird das Modellierungskonzept bewertet.
 
Im Schlussteil wird das Ergebnis festgehalten, ein Fazit getroffen und weiterführende Arbeiten sowie ein Ausblick vorgestellt.
 
\newpage

\section{Grundlagen} \label{sec:section2}
In diesem Kapitel werden zunächst Grundlagen des \ac{iot} erläutert. Anschließend werden die wichtigsten Prozess Modellierungsmethoden dargestellt und Grundlagen des \ac{bpm} erklärt. Zum Abschluss werden zwei Erweiterungen von \ac{bpmn} zur Modellierung von \ac{iot} Workflows vorgestellt.

\subsection{Internet of Things}

Die Idee eines "Internets der Dinge" hat seine Ursprünge in den Konzepten des Anfang der 90er Jahre von Mark Weiser skizzierten "Ubiquitous Computing". 
%Quelle https://www.ics.uci.edu/~corps/phaseii/Weiser-Computer21stCentury-SciAm.pdf
Grundgedanke des Ubiquitous Computing ist eine Erweiterung beliebiger physischer Gegenstände über ihre bestehende Form und Funktion hinaus durch mikroelektronische Komponenten \cite{237456}. Die so entstehenden "smarten" Gegenstände bilden, mit digitaler Logik, Sensorik und der Möglichkeit zur Vernetzung ausgestattet, ein Internet der Dinge. Der Begriff \acl{iot} wurde jedoch erst 1999 von Kevin Ashton im Zusammenhang eines globalen Netzwerks von Objekten welche mit RFID angereicht wurden verwendet. 
Aus technischer Sicht steht hinter dem Internet der Dinge weniger eine einzelne Technologie oder eine spezifische Funktionalität als vielmehr ein Funktionsbündel, welches in seiner Gesamtheit eine neue Qualität der Informationsverarbeitung entstehen lässt.
%Quelle Hot topics-ubiquitous computing

\subsubsection{Smart Objects}
 Zu den charakteristischen Eigenschaften smarter Objekte im erweiterten Internet zählen die in \ref{table:smartObjectsCharacteristics} sichtbaren Eigenschaften

\begin{table}[H]
	\begin{tabularx}{\textwidth}{@{}lL@{}} 
		\toprule
		Charakteristik & Erklärung 
		\\ \midrule
		Identifikation & Objekte im Internet der Dinge sind über ein Nummerierungsschema eindeutig identifizierbar. Die Identifikation ermöglicht die Verknüpfung des Objekts mit Diensten bzw. einem "Datenschatten", d.h. Informationen zu dem Objekt, die auf einem entfernten Server im Netz hinterlegt sind. 
		\\ \hline
		Kommunikation & Im Gegensatz zu herkömmlichen elektronischen Geräten verfügen Objekte im Internet der Dinge über die Möglichkeit zur Vernetzung mit Ressourcen im Netz oder sogar untereinander, um Daten und Dienste gegenseitig zu nutzen. 
		\\ \hline
		Lokalisierung & Smarte Objekte kennen ihren Aufenthaltsort oder sind für andere lokalisierbar, bspw. auf globaler Ebene durch GPS oder in Innenräumen durch Ultraschall 
		\\ \hline
		Speicher & Das Objekt verfügt über Speicherkapazität, so dass es bspw. Informationen über seine Vergangenheit oder Zukunft mit sich tragen kann
		 \\ \hline
		Aktuatorik & Objekte im Internet der Dinge können unter Umständen selbständig Entscheidungen ohne übergeordnete Planungsinstanz treffen, z.B. im Sinne eines Industriecontainers, der seinen Weg durch die Lieferkette selbst bestimmt
		\\ \hline
		Benutzerschnittstelle & Mit dem Aufgehen des Computers im physischen Gegenstand stellen sich auch neue Anforderungen an die Benutzeroberfläche, die meist nicht mehr durch Tasten und Displays realisiert werden kann. Vielmehr braucht es hier neuartige Benutzungsmetaphern analog der Maus und Fenstermetapher graphischer Benutzeroberflächen
		\\ \hline
		\bottomrule
	\end{tabularx}
	\caption{Charakteristiken von smarten Objekten}
	\label{table:smartObjectsCharacteristics}
\end{table}
%Quelle http://www.enzyklopaedie-der-wirtschaftsinformatik.de/lexikon/informationssysteme/lexikon/technologien-methoden/Rechnernetz/Internet/Internet-der-Dinge/index.html

\subsubsection{Domain Model}

\subsection{Prozess Modellierung}

In vielen heutigen Unternehmen unterstützen Informationssysteme nicht mehr nur das Geschäft,sondern sie werden immer mehr zu einem integralen Bestandteil davon. Alle Unternehmen machen einen gewissen Gebrauch von Informationstechnologie, und es ist wichtig, dass ihre Systeme wirklich so aufgebaut sind, dass sie die Unternehmen unterstützen in denen sie zum Einsatz kommen. Das Geschäft bestimmt letztlich die Anforderungen, welche an die Informationssysteme definieren. Die Entwicklung von Software ohne ein angemessenes Verständnis des Kontextes, in welchem diese Software betrieben werden soll, ist nahezu unmöglich. Um ein solches Verständnis zu erlangen, ist es unerlässlich, dass man ein Geschäftsmodell definiert. Ein Modell ist eine vereinfachte Sicht auf eine
komplexe Realität. Diese Abstraktion erlaubt es irrelevante Details zu vernachlässigen und den Fokus auf die Kernelemente zu legen. Effektive Modelle erleichtern zudem
Diskussionen zwischen verschiedenen Stakeholdern im Unternehmen.
Sie ermöglichen es ihnen, sich auf die wichtigsten Grundlagen zu einigen und auf gemeinsame Ziele hinzuarbeiten. Die Modellierung von Geschäftsprozessen ist als Mittel zur Analyse und zum Design von Software akzeptiert und etabliert. Die sich ständig weiterentwickelnden Modelle helfen den Entwicklern auch dabei, ihr Denken zu strukturieren und zu fokussieren. Die Arbeit mit den Modellen dient ihnen zum Verständnis für das Geschäft und erhöht dadurch  das Bewusstsein für neue Möglichkeiten zur Verbesserung des Geschäfts.

\subsubsection{BPMN}

\ac{bpmn} ist ein Standard für die Geschäftsprozessmodellierung, der eine grafische Notation zur Spezifikation von Geschäftsprozessen in einem \ac{bpd} auf Grundlage traditioneller Flussdiagrammtechniken bereitstellt \cite[S.222]{Aagesen2015} 
. Das Ziel von \ac{bpmn} ist es, die Geschäftsprozessmodellierung sowohl für technische Anwender als auch für Geschäftsanwender  zugänglich zu machen. Hierfür wird eine Notation bereitgestellt wird, welche für Geschäftsanwender intuitiv ist und dennoch komplexe Prozesssemantik abbilden kann. Die seit 2011 von der \ac{omg} vorgestellte \ac{bpmn} 2.0-Spezifikation bietet auch Ausführungssemantik sowie das Mapping zwischen den Grafiken der Notation und anderen Ausführungssprachen, insbesondere der \ac{bpel}. \ac{bpmn} ist so konzipiert, dass es für alle Beteiligten leicht verständlich ist.
Zu den Anwendern gehören Business-Analysten, welche die Prozesse erstellen und verfeinern, technische Entwickler, die für die Implementierung zuständig sind sowie Geschäftsleiter welche Prozesse überwachen und verwalten. Im Anhang befindet sich ein Poster mit einer Übersicht über die wichtigsten Modellierungsmethoden von \ac{bpmn}. 
%Quelle: http://www.omg.org/news/whitepapers/Business_Process_Model_and_Notation.pdf

Aufgrund der fehlenden Möglichkeit Flexibilität abzubilden weshalb 2014 von der \ac{omg} ein eigener Standard \ac{cmmn} verabschiedet wurde. Als Case wird eine Aktivität bezeichnet, welche sich nicht einfach wiederholen lässt. Cases sind  von sich entwickelnden Umständen oder von Ad-hoc-Entscheidungen von Wissensarbeitern in Bezug auf  bestimmte Situationen abhängig. Zu den Anwendungsfällen des Case Managements gehören die Lizenzierung und Genehmigung in der Regierung, die Antrags- und Schadensbearbeitung in der Versicherungsbranche, der Patientenversorgung sowie der medizinischen Diagnose im Gesundheitswesen,
Hypothekenbearbeitung im Bankwesen, Problemlösung in Call Centern, Vertriebs- und Betriebsplanung,, Wartung und Reparatur von Maschinen und Anlagen sowie der Konstruktion von Sonderanfertigungen.
%Quelle: http://www.omg.org/spec/CMMN/1.1/PDF

Laut Heise sei die Kombination von \ac{cmmn} und \ac{bpmn} sinnvoll um sowohl strukturierte als auch unstrukturierte Prozesse oder Teilprozesse sinnvoll abbilden zu können.
%Quelle: https://www.heise.de/developer/artikel/Case-Management-und-CMMN-fuer-Entwickler-2569883.html?seite=all

\subsubsection{UML}

 \ac{uml} ist eine grafische Sprache, die die Artefakte verteilter Objektsysteme visualisiert, spezifiziert, konstruiert und dokumentiert \cite{Kleuker}. Es ist der am weitesten verbreitete Standard für Software-Architekten, um Geschäftsanwendungen zu spezifizieren. \ac{uml} wird vor allem für die objektorientierte Softwareentwicklung im Bereich des Software-Engineerings eingesetzt.
Die \ac{uml} wurde in den 90er Jahren als Modellierungssprache und Methodik zur Unterstützung der objektorientierten Programmierung entwickelt. Im Jahr 1997 wurde es als Standard von der \ac{omg} übernommen. Die ersten Versionen 1.X wurden 2005 durch die neu überarbeiteten Versionen 2.X ersetzt. Seit Juni 2015 befindet sich UML in der Version 2.5. Im Zuge dieser Thesis wird \ac{uml} lediglich im Bezug auf Prozessmodellierung mit Aktivitätsdiagrammen verwendet. 


\subsubsection{Geschäftsregeln}

\subsection{BPM}

\subsection{IoT - A}

\subsection{BPMN4CPS}

\newpage

\section{BPM und IoT}
 \textbf{Zusammenfassen}:
Obwohl IoT mittlerweile zum beliebten Schlagwort geworden ist, kämpfen viele immer noch mit dem Konzept, Prozesse auf IoT anzuwenden. BPM, in seinem Kern, nutzt den Workflow, um große Daten- und Informationsmengen zu verwalten, zu aktualisieren und zu verfolgen. Wenn zum Beispiel eine intelligente Uhr Daten von Ihrem Handgelenk empfängt, die sie dann an eine Fitness-Überwachungsapplikation überträgt, woher weiß sie dann, was sie mit diesen Informationen zu tun hat? Speichert es die Daten einfach in seinem Speicher? Schickt es die Daten an andere Anwendungen, die Ihre Gesundheit, Ihre Ernährung und Ihren Zeitplan für medizinische Besuche überwachen? Prozesse wie diese nutzen BPM, um intelligente Objekte und Anwendungen in der richtigen Reihenfolge zu verwalten. Das Volumen der Daten, mit denen wir täglich arbeiten, nimmt exponentiell zu, so dass es für uns eine absolute Notwendigkeit ist, diese Informationen mit Prozessen besser zu verwalten, die uns gut dienen. BPM erhöht den Wert des IoT durch die Verbindung von intelligenten Objekten. Dies wiederum erweitert ihre Integration und Orchestrierung. Da immer mehr Geräte angeschlossen werden und die IoT-Nutzung zunimmt, besteht eine größere Wahrscheinlichkeit für Chaos durch die schnelle Multiplikation einzelner Schnittstellen. Intelligente Geräte senden Informationen von einem angeschlossenen Gerät über eine API, um eine Antwort zu aktivieren. Die Antworten können in einem kompletten Prozess orchestriert werden, wobei nachfolgende Antworten wie z.B. das Öffnen einer Autotür oder der systematische Zugriff auf bestimmte Dateien aufgerufen werden. Die Integration von BPM ermöglicht es, viele Dinge innerhalb des IoT richtig und nahtlos zusammen zu managen.

\subsection{Typische Muster und Best Practices von IoT Workflows}

\subsection{Unterschiede IoT Workflows zu regulären Workflows}

\subsection{Evaluierungskriterien}

\subsection{Bewertung der Modellierungsmethoden}

\subsection{Modellierungskonzept}

\newpage

\section{Evaluierung}

\newpage

\section{Schlussteil}

\subsection{Ergebnis}

\subsection{Fazit}

\subsection{Weiterführende Arbeit/ Ausblick}

\newpage

\addsec{Literaturverzeichnis}
\printbibliography[heading=none]
\newpage
\addsec{Anhang}
\subsection*{Unterbereich Anhang}
\end{document}
