%% LaTeX2e file `Thesis-bpmn4cpsbewertung.tex'
%% generated by the `filecontents' environment
%% from source `Thesis' on 2018/02/21.
%%
 \begin{longtable}{P{4cm}|X}
  \caption{Umsetzung der IoT spezifischen Anforderungen durch das BPMN4CPS Modellierungskonzept}\\
  \label{table:evaluierungskriterien}
  % Definition des ersten Tabellenkopfes auf der ersten Seite
  \textbf{Anforderung} & \textbf{Erklärung}   \\ \hline
  \endfirsthead % Erster Kopf zu Ende
  %  Definition des Tabellenkopfes auf den folgenden Seiten
  \textbf{Anforderung} & \textbf{Erklärung}  \\ \hline
  \endhead % Zweiter Kopf ist zu Ende
  % Ab hier kommt der Inhalt der Tabelle
  Device als Akteur & In \ac{bpmn4cps} werden Devices nicht als eigene Lane dargestellt, da dies aber bei einer großen Anzahl an Lanes führen würde sondern wird über eine farbliche Zuweisung dargestellt. Devices lassen sich über Parameter eindeutig physischen Dingen zuweisen. \\ \hline
  Physische Dinge  & Die \ac{bpmn4cps} spezifischen Aufgabentypen beziehen sich direkt auf die physischen Dinge und überwachen beziehungsweise verändern dessen Zustand. Die physische Entität selbst wird als Prozess Teilnehmer dargestellt und ist mit dem physischen Prozess Pool verbunden.\\ \hline
  Spezielle Aufgabentypen & \ac{bpmn4cps} Konzept bietet zum einen Physical Tasks welche in Actuators Task und Sensing Task unterschieden wird und zum Anderen Cyber Tasks die sich in WebServiceTasks beziehungsweise CloudServiceTasks sowie EmbeddedService Tasks unterscheiden.\\ \hline
  Informationen  & Das \ac{bpmn4cps} Konzept verwendet für die von Devices erzeugten Informationen lediglich die standard \ac{bpmn} Elemente.\\ \hline
  Mobiltät & Devices können über den Parameter isMovable als mobil gekennzeichnet werden, besitzen allerdings keinen optischen Indikator dafür. Tasks sind besitzen weder Parameter über die Mobilität noch lassen sich diese ortsabhängig ausführen.\\ \hline
  Granularität & Durch die klassische \ac{bpmn} Elemente lassen sich Prozesse aus beliebig vielen geringer komplexen Subprocesses zusammenstellen.\\
 \end{longtable}
