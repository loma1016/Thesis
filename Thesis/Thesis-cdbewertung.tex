%% LaTeX2e file `Thesis-cdbewertung.tex'
%% generated by the `filecontents' environment
%% from source `Thesis' on 2018/02/21.
%%
 \begin{longtable}{P{4cm}|X}
  \caption{Umsetzung der IoT spezifischen Anforderungen durch BPMN Choreographiediagramme}\\
  \label{table:evaluierungskriterien}
  % Definition des ersten Tabellenkopfes auf der ersten Seite
  \textbf{Anforderung} & \textbf{Erklärung}   \\
  \hline
  \endfirsthead % Erster Kopf zu Ende
  %  Definition des Tabellenkopfes auf den folgenden Seiten
  \textbf{Anforderung} & \textbf{Erklärung}  \\
  \hline
  \endhead % Zweiter Kopf ist zu Ende
  % Ab hier kommt der Inhalt der Tabelle
  Device als Akteur & Aufgaben in Choreographiediagrammen besitzen immer einen initiierenden Teilnehmer und mindestens einen weiteren Teilnehmer. Je nachdem in welcher Rolle, also aktiver oder passiver Teilnehmer das Device agiert kann es sich dadurch in beiden Fällen darstellen lassen.\\ \hline
  Physische Dinge  & Physische Dinge lassen sich im Choreographiediagramm als passiver Prozessteilnehmer abbilden. Allerdings wird die Zugehörigkeit zwischen Device und physischen Ding hier nicht berücksichtigt und eine Zuweisung anhand der graphischen Notation ist nicht möglich. Der \ac{iot} spezifische Zusammenhang wird hierbei also nicht deutlich\\ \hline
  Spezielle Aufgabentypen & Choreographiediagramme besitzen keine besonderen Aufgabentypen. Die \ac{iot} spezifischen Aufgabentypen lassen sich dadurch also nicht darfstellen. \\ \hline
  Informationen  &  Informationen lassen sich nur in Form von Nachrichten darstellen. Diese Darstellung wird den Anforderungen nicht gerecht da die \ac{iot} spezifischen Merkmale nicht berücksichtig werden.\\ \hline
  Mobiltät & Es gibt keine Möglichkeit Prozessteilnehmer als mobil beziehungsweise Aufgaben als standortabhängig darzustellen.\\ \hline
  Granularität & Jeder Prozess kann aus einer Vielzahl von Unterprozesse dargestellt werden.\\
 \end{longtable}
