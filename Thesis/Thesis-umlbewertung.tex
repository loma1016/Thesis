%% LaTeX2e file `Thesis-umlbewertung.tex'
%% generated by the `filecontents' environment
%% from source `Thesis' on 2018/02/21.
%%
 \begin{longtable}{P{4cm}|X}
  \caption{Umsetzung der IoT spezifischen Anforderungen durch UML Aktivitätsdiagramme}\\
  \label{table:evaluierungskriterien}
  % Definition des ersten Tabellenkopfes auf der ersten Seite
  \textbf{Anforderung} & \textbf{Erklärung}   \\ \hline
  \endfirsthead % Erster Kopf zu Ende
  %  Definition des Tabellenkopfes auf den folgenden Seiten
  \textbf{Anforderung} & \textbf{Erklärung}  \\ \hline
  \endhead % Zweiter Kopf ist zu Ende
  % Ab hier kommt der Inhalt der Tabelle
  Device als Akteur & In \ac{uml} Aktivitätsdiagrammen lassen sich Devices als Swimlanes darstellen und bieten somit eine Gruppierung von Knoten also Aufgaben, welche vom Device übernommen werden.\\ \hline
  Physische Dinge  & Physische Dinge können in Aktivitätsdiagrammen als Objektknoten dargestellt werden deren Zustand durch Aktionsknoten verändert wird. Eine Darstellung als eigene Lane wie es in \ac{bpmn} existiert ist nicht möglich da hier keine zugeklappte Lane existiert und das Physische Objekt an sich keine eigenen Aufgaben übernimmt.\\ \hline
  Spezielle Aufgabentypen & \ac{uml} Aktivitätsdiagramme besitzen keine besondere Form von Aufgaben sondern besitzen lediglich eine Darstellung durch Aktionsknoten. \\ \hline
  Informationen  & Informationen werden klassischer Weise als Objektknoten modelliert, diese können beliebig viele Attribute enthalten. Eine Modellierung von Informationen mit dem selben graphischen Element wie das physische Ding würde das Modell jedoch \\ \hline
  Mobiltät & UML Aktivitätsdiagramme bieten keine Weise die Mobilität von Prozessteilnehmern oder die Ortsabhängigkeit von Aktionen darzustellen.\\ \hline
  Granularität & Prozesse könne aus mehreren Unterprozessen zusammengesetzt werden.\\
 \end{longtable}
