%% LaTeX2e file `Thesis-bpmnbewertung.tex'
%% generated by the `filecontents' environment
%% from source `Thesis' on 2018/02/16.
%%
 \begin{longtable}{p{4.5cm}X}
  \caption{Umsetzung der IoT spezifischen Anforderungen von BPMN}\\
  \label{table:evaluierungskriterien}
  % Definition des ersten Tabellenkopfes auf der ersten Seite
  \textbf{Anforderung} & \textbf{Erklärung}   \\
  \hline
  \endfirsthead % Erster Kopf zu Ende
  %  Definition des Tabellenkopfes auf den folgenden Seiten
  \textbf{Anforderung} & \textbf{Erklärung}  \\
  \hline
  \endhead % Zweiter Kopf ist zu Ende
  siehe n{\"a}chste Seite\\
  \endfoot
  \hline
  Tabelle zu Ende \\
  \endlastfoot
  % Ab hier kommt der Inhalt der Tabelle
  Device als Akteuer & Devices lassen sich in \ac{bpmn} als Lane darstellen und sind in der Lage Tätigkeiten auszuführen.
  \\ \hline
  Phyische Dinge  & Physische Dinge lassen sich als zugeklappter Pool darstellen. Die Zuweisung von Device zu physischem Ding lässt sich jedoch nur graphisch darstellen.
  \\ \hline
  Spezielle Aufgabentypen & Für die \ac{iot} spezifischen Aufgaben der Sensorik und Aktuatorik besitzt \ac{bpmn} keine geeigneten Elemente
  \\ \hline
  Informationen  & Die Informationen welche durch \ac{iot} Devices generiert werden können in \ac{bpmn} als DataObject dargestellt werden diese geben allerdings keine Auskunft über ihre Qualität, Herkunft oder Zeitpunkt der Erstellung.
  \\ \hline
  Mobiltät & \ac{bpmn} bietet keine Möglichkeit Prozessteilnehmer oder Aufgaben als mobil oder Standortabhängig zu kennzeichnen.
  \\ \hline
  Granularität & Jeder Prozess kann aus einer Vielzahl von Subprocesses dargestellt werden.
  \\ \hline
 \end{longtable}
