%% LaTeX2e file `Thesis-Evaluierung.tex'
%% generated by the `filecontents' environment
%% from source `Thesis' on 2018/02/21.
%%
 \begin{longtable}{| l |c|c|c|c|c|c|}
  \caption{Bewertung der Modellierungsmethoden}\\ \hline
  \label{table:Evaluierung}
  % Definition des ersten Tabellenkopfes auf der ersten Seite
   & \textbf{IoT - A} & \textbf{BPMN4CPS} & \textbf{PD}\footnotemark[1] & \textbf{CD}\footnotemark[2] & \textbf{AD}\footnotemark[3] &\textbf{eEPK}  \\\hline
  \endfirsthead % Erster Kopf zu Ende

  %  Definition des Tabellenkopfes auf den folgenden Seiten
  & \textbf{IoT - A} & \textbf{BPMN4CPS} & \textbf{PD}\footnotemark[1] & \textbf{CD}\footnotemark[2] & \textbf{AD}\footnotemark[3] &\textbf{eEPK} \\ \hline
  \endhead % Zweiter Kopf ist zu Ende
  \multicolumn{7}{P{10.8cm}}{\textit{+ = Anforderung erfüllt, 0 = Anforderung teilweise erfüllt, - = Anforderung nicht erfüllt}}\\
  \endlastfoot


  % Ab hier kommt der Inhalt der Tabelle
  Device als Akteur & +  & + & 0 & 0 & 0 & 0 \\ \hline

  Physische Dinge  & + & + & 0 & 0 & 0 & - \\ \hline

  Spezielle Aufgabentypen & + & + & - & - & -& - \\ \hline

  Informationen  & + & - & - & - & - & - \\ \hline

  Mobilität & 0 & - & - & - & - & 0 \\ \hline

  Granularität & + & + & + & + & + & - \\ \hline

 \end{longtable}
