%% LaTeX2e file `Thesis-iotabewertung.tex'
%% generated by the `filecontents' environment
%% from source `Thesis' on 2018/02/21.
%%
 \begin{longtable}{P{4cm}|X}
  \caption{Umsetzung der IoT spezifischen Anforderungen durch das IoT-A Modellierungskonzept}\\
  \label{table:evaluierungskriterien}
  % Definition des ersten Tabellenkopfes auf der ersten Seite
  \textbf{Anforderung} & \textbf{Erklärung}   \\ \hline
  \endfirsthead % Erster Kopf zu Ende
  %  Definition des Tabellenkopfes auf den folgenden Seiten
  \textbf{Anforderung} & \textbf{Erklärung}  \\ \hline
  \endhead
  % Ab hier kommt der Inhalt der Tabelle
  Device als Akteur & Im \ac{iota}Modellierungskonzept wird \ac{bpmn} um die Rolle eine IoT Devices erweitert, dieser besitzt Parameter welche die Zuweisung von Devices zu \ac{eoi} ermöglicht.\\ \hline
  Physische Dinge  & Genauso wie Devices besitzen \ac{eoi} im \ac{iota} Konzept Parameter welche die Zuweisung zu Device und generierten Informationen ermöglicht.\\ \hline
  Spezielle Aufgabentypen & Mit der Einführung von Actuation Task und Sensing Task werden die grundlegendsten Funktionsweisen von \ac{iot} Devices abgedeckt.\\ \hline
  Informationen  & Die von den \ac{iot} Devices bereitgestellten Informationen werden durch eine Erweiterung von der vorhanden DataObjects in \ac{bpmn} umgesetzt. Diese besitzen Metadaten über ihre Herkunft, Erstellungsort und Zeit sowie ihrer Qualität.\\ \hline
  Mobiltät & Prozessteilnehmer und Aufgaben können durch die Einführung der Mobile Property beziehungsweise Location-based Activity als mobil gekennzeichnet werden. Allerdings handelt es sich hierbei eher um eine kosmetische Erweiterung als um eine funktionale. Eine ortsabhängige Ausführung von Task wird durch die Erweiterung nicht gewährleistet.\\ \hline
  Granularität & Durch die klassische \ac{bpmn} Elemente lassen sich Prozesse aus beliebig vielen geringer komplexen Subprocesses zusammenstellen.\\
 \end{longtable}
