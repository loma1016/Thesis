%% LaTeX2e file `Thesis-pdbewertung.tex'
%% generated by the `filecontents' environment
%% from source `Thesis' on 2018/02/21.
%%
 \begin{longtable}{P{4cm}|X}
  \caption{Umsetzung der IoT spezifischen Anforderungen durch BPMN Prozessdiagramme}\\
  \label{table:evaluierungskriterien}
  % Definition des ersten Tabellenkopfes auf der ersten Seite
  \textbf{Anforderung} & \textbf{Erklärung}   \\ \hline
  \endfirsthead % Erster Kopf zu Ende
  %  Definition des Tabellenkopfes auf den folgenden Seiten
  \textbf{Anforderung} & \textbf{Erklärung}  \\ \hline
  \endhead % Zweiter Kopf ist zu Ende
  % Ab hier kommt der Inhalt der Tabelle
  Device als Akteur & Devices lassen sich in \ac{bpmn} Prozessdiagrammen als Lane darstellen und sind in der Lage Tätigkeiten auszuführen.\\ \hline
  Physische Dinge  & Physische Dinge lassen sich als zugeklappter Pool darstellen. Die Zuweisung von Device zu physischem Ding lässt sich jedoch nur graphisch darstellen.\\ \hline
  Spezielle Aufgabentypen & Für die \ac{iot} spezifischen Aufgaben der Sensorik und Aktuatorik besitzt \ac{bpmn} keine geeigneten Elemente\\ \hline
  Informationen  & Die Informationen welche durch \ac{iot} Devices generiert werden können in \ac{bpmn} Prozessdiagrammen als DataObject dargestellt werden diese geben allerdings keine Auskunft über ihre Qualität, Herkunft oder Zeitpunkt der Erstellung. \\ \hline
  Mobiltät & \ac{bpmn} Prozessdiagramme bieten keine Möglichkeit Prozessteilnehmer oder Aufgaben als mobil oder standortabhängig zu kennzeichnen.\\ \hline
  Granularität & Jeder Prozess kann aus einer Vielzahl von Unterprozessen dargestellt werden.\\
 \end{longtable}
