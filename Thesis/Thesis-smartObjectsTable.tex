%% LaTeX2e file `Thesis-smartObjectsTable.tex'
%% generated by the `filecontents' environment
%% from source `Thesis' on 2018/02/07.
%%
 \begin{longtable}{p{4.5cm}X}
  \caption{Charakteristiken von "smarten" Objekten\cite{iotwiki}}\\
  \label{table:smartObjectsCharacteristics}
  % Definition des ersten Tabellenkopfes auf der ersten Seite
  \textbf{Charakteristik} & \textbf{Erklärung}   \\
  \hline
  \endfirsthead % Erster Kopf zu Ende
  %  Definition des Tabellenkopfes auf den folgenden Seiten
  \textbf{Charakteristik} & \textbf{Erklärung}  \\
  \hline
  \endhead % Zweiter Kopf ist zu Ende
  Weiter auf der n{\"a}chste Seite\\
  \endfoot
  \hline
  Tabelle zu Ende \\
  \endlastfoot
  % Ab hier kommt der Inhalt der Tabelle
  Identifikation & Objekte im Internet der Dinge sind über einen Schlüssel eindeutig identifizierbar. Diese Identifikation ermöglicht die Verknüpfung des Objekts mit Diensten, welche Informationen des physischen Objektes auf einem Server bereitstellen.
  \\ \hline
  Kommunikation & Im Gegensatz zu herkömmlichen phyischen Objekten verfügen Objekte im Internet der Dinge über die Möglichkeit Ressourcen im Netz oder sogar untereinander zur Verfügung zu stellen, um Daten und Dienste gegenseitig zu nutzen.
  \\ \hline
  Sensorik & Das "smarte" Objekt sammelt Informationen über seine Umwelt (Temperatur, Lichtverhältnisse, Luftdruck usw.), zeichnet diese auf und/oder reagiert darauf
  \\ \hline
  Lokalisierung & Smarte Objekte kennen ihren Aufenthaltsort oder sind für andere lokalisierbar, bspw. auf globaler Ebene durch GPS oder in Innenräumen durch Ultraschall
  \\ \hline
  Speicher & Das Objekt verfügt über Speicherkapazität, so dass es beispielsweise. Informationen über seine Vergangenheit mit sich tragen kann
  \\ \hline
  Aktuatorik & Objekte im Internet der Dinge können unter Umständen selbständig Entscheidungen ohne übergeordnete Planungsinstanz treffen, z.B. im Sinne eines Industriecontainers, der seinen Weg durch die Lieferkette selbst bestimmt
  \\ \hline
  Benutzerschnittstelle & Mit dem Aufgehen des Computers im physischen Gegenstand stellen sich auch neue Anforderungen an die Benutzeroberfläche, die meist nicht mehr durch Tasten und Displays realisiert werden kann. Vielmehr braucht es hier neuartige Benutzungsmetaphern analog der Maus und Fenstermetapher graphischer Benutzeroberflächen
  \\ \hline
 \end{longtable}
