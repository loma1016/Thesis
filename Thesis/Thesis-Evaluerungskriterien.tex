%% LaTeX2e file `Thesis-Evaluerungskriterien.tex'
%% generated by the `filecontents' environment
%% from source `Thesis' on 2018/02/21.
%%
 \begin{longtable}{P{4.5cm}|X}
  \caption{Anforderung an IoT Modellierung}\\
  \label{table:evaluierungskriterien}
  % Definition des ersten Tabellenkopfes auf der ersten Seite
  \textbf{Anforderung} & \textbf{Erklärung}   \\ \hline
  \endfirsthead % Erster Kopf zu Ende
  %  Definition des Tabellenkopfes auf den folgenden Seiten
  \textbf{Anforderung} & \textbf{Erklärung}  \\ \hline
  \endhead % Zweiter Kopf ist zu Ende
  % Ab hier kommt der Inhalt der Tabelle
Device als Akteur & Devices müssen als neuer Akteur im Geschäftsprozess modellierbar sein.\\ \hline
Physische Dinge  & Informationen, Devices und Services müssen eindeutig physischen Dingen zuweißbar sein.\\ \hline
Spezielle Aufgabentypen & IoT Devices als Prozessteilnehmer benötigen eigene Aufgabentypen um ihren charakteristischen Merkmalen gerecht zu werden.\\ \hline
Informationen  & Die Informationen welche durch IoT Devices bereitgestellt werden müssen Metadaten über den Zeitpunkt, Ort, Herkunft ihrer Erstellung sowie über ihre Qualität besitzen.\\ \hline
Mobiltät & Prozessteilnehmer und Aktivitäten in IoT Prozessen sind ortsabhängig. Diese Abhängigkeit muss modellierbar sein.\\ \hline
Granularität & Prozesse müssen deutlich in mehrere Unterprozesse aufteilbar sein.\\
 \end{longtable}
